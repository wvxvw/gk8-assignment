% Created 2021-05-04 Tue 23:10
% Intended LaTeX compiler: pdflatex
\documentclass[11pt]{article}
\usepackage[utf8]{inputenc}
\usepackage[T1]{fontenc}
\usepackage{graphicx}
\usepackage{grffile}
\usepackage{longtable}
\usepackage{wrapfig}
\usepackage{rotating}
\usepackage[normalem]{ulem}
\usepackage{amsmath}
\usepackage{textcomp}
\usepackage{amssymb}
\usepackage{capt-of}
\usepackage{hyperref}
\author{wvxvw}
\date{\today}
\title{}
\hypersetup{
 pdfauthor={wvxvw},
 pdftitle={},
 pdfkeywords={},
 pdfsubject={},
 pdfcreator={Emacs 26.3 (Org mode 9.1.9)}, 
 pdflang={English}}
\begin{document}


\section{Describe the architecture of a possible automation testing system}
\label{sec:orgf024ab3}
There are different aspects of software that are typically subjected
to testing.  The categories are roughly as follows:
\begin{itemize}
\item Functional testing.  This ensures that the product matches
specification.
\item Performance testing.  This ensures that the product doesn't
exhibit pathological behavior when run over long periods of time.
In addition to above, such tests usually try to establish that the
product adequately responds to change in pressure from resources
or input.
\item Benchmarking.  This tries to establish how well the product can
utilize available resources.
\end{itemize}

Not only all of the above require different approaches, these
aspects have further sub-division, that, in practice, often leads to
different, unconnected, implementations.  For example, functional
testing is often envisioned as a hierarchy of scope of a single test:
\begin{itemize}
\item Unit testing.  This deals with the essential and indivisible
building blocks of the software being tested.  Such tests are
designed to be executed by software developers w/o the aid from
Q/A as part of their development cycle.
\item Acceptance testing.  This is usually a small group of tests that
is specifically designed to cover as many features of the product
in as little operations as possible.  The goal of such testing is
to serve as a gatekeeper for Q/A, as often times sending a product
to Q/A is expensive, while "dead on delivery" software will
produce very little value proportionate to the effort spent.  This
sort of testing is often dubbed as "sanity" testing.  Such tests
are designed to run whenever a features is merged into a common
code repository.
\item Nightly (alternatively: periodical testing).  This testing is
designed to exhaustively search for defects in software, a
separate part of periodic testing is regression testing, which
requires its own treatment.  This testing is the bread-and-butter
of Q/A, as this is where most non-trivial bugs are found.  Another
aspect of periodic testing is that it usually tries to test the
integration between multiple components, where end-to-end testing
is a special case of trying to engineer complete usage scenarios.
Yet another aspect of periodic testing is "error injection",
i.e. tests that examine system's behavior in situations where
hardware components fail, or software components outside the
system fail / provide faulty input.

In the light of the above, it's very optimistic to hope for \emph{a}
testing system.  There aren't real testing systems that attempt to
cover all the aspects of software testing, and it's hard to
imagine what such system would be like, should anyone attempt
making one.

This is why I will interpret this task as asking for a design for
some aspect of testing.  I will choose end-to-end testing, as it
is my strong conviction that of all kinds of functional testing,
end-to-end provides strongest guarantees (e.g. the product is
shown to work at least some times), while other aspects cannot
promise even as much.
\end{itemize}

\subsection{End-to-end testing for BitTorrent client}
\label{sec:org68fef3a}
\subsubsection{Which frameworks would you use to write the automation?}
\label{sec:org6f25a94}
I'm not aware of existing testing frameworks that go beyond
unit-testing.  Unit testing received a lot of attention from the
programming community, while other aspects of testing have very
sketchy coverage.  In my experience, all systems that purported to
serve this goal were in-house development, that never transferred
from product to product.  Perhaps, a general system could be
created for such tests, but this would be clearly beyond the scope
of this exercise.

\subsubsection{Which tools would you use to run the BitTorrent that you're testing?}
\label{sec:orgebd937f}
I assume that I'm testing a BitTorrent client, and that the
functionality of, for example, trackers is out of scope for this
exercise.

The way I'd approach this is by adopting and extending \texttt{etorrent}
client \url{https://github.com/jlouis/etorrent} .  Erlang is chosen for
the provisions the language runtime makes for distributed
computation, the embedded relational database, minimal
requirements for installation and a good track record of creating
testing frameworks (eg. QuickCheck).

Using \texttt{Erlang's} OTP would allow me to abstract the deployment
process from the actual hardware (or virtual hardware) on which
the tests should run, I would also be able to use presets to
configure networks, as instead of creating custom networking layer
for tests, in essence, I'd be configuring the networking layer for
OTP.

I would also need access to the version control tool used by the
development.  I'll assume \texttt{Git} is used.

I would also need access to bug tracker software used in the
project.  I'll assume \texttt{JIRA} is used.

I would also need a configuration tool to set up multi-host
environments.  I'll assume \texttt{Terraform} is used.

Depending on company's financial situation (in general, public
cloud services aren't cheap, but it's possible to get exclusive
contracts that mitigate the costs), I might thus use
vendor-specific software to connect to cloud services, or use
on-premises lab equipment if such is available.

To speed up development process, and since hardware emulation is
hardly essential to BitTorrent client, I'd use some container
technology to simulate multi-host networks on single physical
host.  \texttt{Docker} seems like a popular choice here.

Finally, I'd need a test runner to deal with the test maintenance
issues, s.a. sending alerts, keeping logs and artifacts, managing
test runs and providing interface for developers to execute tests
and evaluate test results.  There are plenty of projects that try
to fill this niche.  Unfortunately, none are any good.  \texttt{Jenkins},
however, is a popular choice.

\subsubsection{When should the automation project run?}
\label{sec:orge24e6dc}
If end-to-end tests take less than 24 hours, then they should run
daily, if they take less than two days, then bi-daily, and so on.
This is, of course, subject to financial constraints.

\subsubsection{Create a simple sketch of the framework, naming all its parts (binaries/machines)}
\label{sec:org1ee0a52}
\begin{description}
\item[{Deployer}] The test module responsible for deploying SUT
(system under test) and the test itself.  This is
the module that contains \texttt{Terraform} scripts and
\texttt{Docker} images necessary to create test assets.
\item[{Runner}] The test module responsible for execution of the test
plan.  It should receive input from \texttt{Deployer}
describing the SUT, and organize tests in such a way
that they use given system resources.
\item[{Analyzer}] The test module responsible for collection of test
results, their refinement and shipment to persistent
storage.  This module is also responsible for
providing initial triage and RCA (root cause
analysis), as well as alerts.
\item[{Interface}] The test module responsible for implementing the
client side of the interface with the SUT.  This
module provides functions and other definitions
used across different tests to perform actions on
the system.  This module also contains the model of
the system as well as the code to assess that the
system is in acceptable state.
\item[{Harness}] The test module to deal with the administrative side
of tests: the connection with \texttt{JIRA} and \texttt{Git}.  This
module should be able to, given input from developer,
select the tests relevant to the feature being
developed, filter the tests that are known to fail
due to known bugs, update reporting systems, initiate
code reviews, or prevent merges.  This module is also
responsible for providing testing primitives
s.a. \texttt{suite}, \texttt{scenario}, \texttt{step}.
\item[{Individual Test Modules}] For example, \texttt{E2E} module would be
responsible for describing scenarios developed by testers to
that end.  Testers would be thus responsible to use the
functions and other definitions defined in \texttt{Harness} module
to structure their tests and functions and definitions from
\texttt{Interface} module to interface with the SUT.
\end{description}

\section{Create a Software Test Description (STD) Document, sorting tests by section/scenario}
\label{sec:org0314c1e}
While story / scenario approach is popular in the industry today, I
don't believe in it.  My argument is that such approach provides
indeterminate coverage and creates a lot of busy work for Q/A.  Not
only busy work is a problem in itself, it also creates an
environment that welcomes people of low aptitude.  This creates a
vicious circle, whereby testers are treated as less capable
programmers, while programmers, on the other hand, will never
actively participate in testing.  This results in relative
uselessness of tests, where programmers, knowing the tests to be
mostly useless will implement ad hoc, poorly supported and poorly
documented framework to cover their individual needs wrt' testing.

I'm a proponent and supporter of model-based testing, where the
objective of the tester is to design a model of the SUT, while
describing as precisely as possible the properties of the SUT on one
hand, while, on the other hand, the testers need to implement the
runner that systematically and automatically verifies the properties
of the system.  So-called "fuzzy" testing or "property-based"
testing are thus steps in the right direction.

However, I also understand that it's hard to convince people to do
things in an unusual way.  It is also hard to find Q/A professionals
skilled in this approach, this is why I'm going to describe the
testing plan in a more traditional way to the best of my ability.

\subsubsection{Make sure you mention ALL the necessary sections}
\label{sec:org89e698d}
As I already mentioned before, story / scenario approach isn't the
one that can provide exhaustive testing.  Not only that, it's
impossible to even tell what fraction of functionality is being
covered.

\subsubsection{You're NOT required to write the steps of each scenario, but make sure you explain the major scenarios in each section.}
\label{sec:orgebad890}
I will use Gherkin to give examples of possible scenarios.
Gherkin is a language invented by Ruby TDD televangelists.  I'm
not affiliated with them, but it might be a convenient common
ground.

\begin{verbatim}
Feature: Publish To Tracker

  Scenario: Publish To Multiple Trackers
    Given I deploy the system
      | image       | role    | id | location  | platform |
      | tracker.ova | tracker | t1 | us-east-2 | aws      |
      | tracker.ova | tracker | t2 | us-west-2 | aws      |
      | client.ova  | client  | c1 | us-west-1 | aws      |
      | client.ova  | client  | c2 | us-east-1 | aws      |
    When client "c1" publishes to trackers "t1,t2" file "test"
    Then client "c2" merges download from trackers "t1,t2" of file "test"

  Scenario: Handle Network Partition
    Given I deploy the system
      | image       | role    | id | location  | platform |
      | tracker.ova | tracker | t1 | us-east-2 | aws      |
      | client.ova  | client  | c1 | us-west-1 | aws      |
      | client.ova  | client  | c2 | us-east-1 | aws      |
    When client "c1" publishes to trackers "t1" file "test"
    And client "c2" download file "test"
    And client "c1" starts downloading file "test"
    And client "c2" is stopped
    Then client "c1" has peer list
      |  id |
\end{verbatim}

Obviously, there are countless many possible scenarios like the
ones shown in example above.  For instance, you can indefinitely
expand the last scenario by bringing \texttt{c2} online and sending it
back offline, thus producing, while not very meaningful,
definitely different tests.

Similarly, you can, for example, add more trackers to see if
downloads are merged for more and more trackers.  Not very useful,
but, definitely different.

\section{Without ever touching the BitTorent system, please describe 2-3 potential bugs that are likely to be found the system?}
\label{sec:org300f17e}
I'm not sure what does "ever touching" mean in this context.  Also,
the sentence is missing a preposition before "system".  Also, I
belive that "BitTorent" is a typo, should've been "BitTorrent".
I'll assume that "without ever touching" means that I should pretend
I know nothing about BitTorrent (I know very little indeed, so
that's not hard).  I'll assume that the missing preposition is "in",
i.e. bugs are found in the system.

There are indeed many problems virtually any program may encounter,
which are unrelated to the nature of BitTorrent protocol / system.
For instance, a program will usually take some arguments when
started, on UNIX systems arguments come from two sources: supplied
through environment, and supplied through command line.  Some
trivial failures will include handling Unicode or non-Unicode input.

Another trivial and general thing to check is whether a program
responds well to running multiple copies of it. I.e. try starting
another copy of the program once one copy is executing.  Some system
resources are unique, sockets, locks, CPU cores etc.  When the
program isn't designed to deal with other copies of itself it may
crash or stall instead of informing the user the action cannot be
performed.

Yet another trivial problem is the one of permissions.  Some
programs may assume free access to system utilities which might not
be available due to user policies on a particular system.  Thus it's
useful to try running the program as administrator, or as a
unprivileged user.

It is hard, however, to find useful tests that are general enough
not to require any specific knowledge of the SUT.
\end{document}
